\documentclass[11pt]{article}
\usepackage{graphicx}
\usepackage{hyperref}
\usepackage{booktabs}
\usepackage{geometry}
\geometry{margin=1in}

\title{Project 12: Hallucination in Generative AI Code Outputs: Detection and Mitigation}
\author{Your Name \\ Student ID}
\date{\today}

\begin{document}
\maketitle

\begin{abstract}
This project investigates how generative AI models produce incorrect or nonexistent code logic (hallucinations). We analyze failure patterns, propose mitigation techniques, and validate their effectiveness across multiple models.
\end{abstract}

\section{Introduction}
Generative AI models can produce code that looks correct but fails functional specifications. This study compares models including Code LLaMA, StarCoder, and LLaMA variants.

\section{Dataset}
The dataset consists of programming tasks in a strictly defined JSON format (see data/tasks.json).

\section{Methodology}
\subsection{Models Used}
\begin{itemize}
    \item Code LLaMA
    \item StarCoder
    \item LLaMA variants
\end{itemize}

\subsection{Detection of Hallucinations}
We detect hallucinated patterns by comparing generated outputs against expected outputs and test cases.

\section{Root Cause Analysis}
We categorize hallucinations by type (control flow errors, incorrect logic, missing edge-case handling).

\section{Mitigation Strategies}
\begin{itemize}
    \item Test-driven filtering
    \item Rule-based output repair
\end{itemize}

\section{Results}
Figures and tables summarize error rates and mitigation results.

\begin{figure}[h]
    \centering
    \includegraphics[width=0.8\textwidth]{figures/hallucination_pattern.png}
    \caption{Example Hallucination Pattern}
\end{figure}

\section{Conclusion}
Summarizes findings and future work.

\bibliographystyle{plain}
\bibliography{references}

\end{document}
